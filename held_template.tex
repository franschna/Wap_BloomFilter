\documentclass[pdftex,
               12pt,
               DIV=12,
               a4paper,
               twoside,
               parskip=half,
               abstract=true]{scrartcl}

\usepackage[american,naustrian]{babel}
\usepackage[utf8]{inputenc}
\usepackage[T1]{fontenc}
\usepackage{lineno}
  \linenumbers

\usepackage{graphicx}
\usepackage{amsmath}
\usepackage{amsfonts}
\usepackage{amssymb}
\usepackage{hyperref}
\usepackage{csquotes}
\usepackage[numbers]{natbib}
\usepackage{float}
\usepackage{algorithm}
\usepackage{algpseudocode}


% Info zu Autoren und Titel
\author{Jambura Anna, Pürstinger Kathrin,\\ Schnappauf Franziska, Thiele Coco\\}
\title{Bloom Filter}

\begin{document}
\maketitle

\section{Grundlagen und Motivation}
\subsection{Das Membership-Problem}
Seien ein beliebiges Element $x$ und eine Menge $S$ gegeben. Das Membership-Problem ist eine Bezeichnung für die Fragestellung: „Ist das Element $x$ Teil der Menge $S$?“ Diese Frage tritt in vielen verschiedenen Bereichen und Anwendungen auf. Einige Beispiele dafür sind Datenbankenabfragen und URL-Caching in Web-Browsern.
Klassische Ansätze, wie Listen, Hashtabellen oder Suchbäume liefern eine exakte Antwort auf die Frage, jedoch benötigen sie alle entsprechend viel Zeit und Speicherplatz.

\subsection{Lösungsansatz}
Bloomfilter wurden 1970 von Burton H. Bloom entwickelt, um den hohen Ressourcenbedarf zu umgehen. Sie sind probabilistische Datenstrukturen, das bedeutet sie arbeiten mit Wahrscheinlichkeiten anstatt absoluter Sicherheit. 

Dabei erlauben sie False-Positives in einem begrenzten Ausmaß. Ein Filter kann also fälschlicherweise melden, das Element $x$ sei Teil der Menge $S$, auch wenn dies nicht der Fall ist. Umgekehrt sind False-Negatives jedoch ausgeschlossen. Wenn $x$ tatsächlich ein Element von $S$ ist, wird das der Filter immer korrekt erkennen. Mit anderen Worten: Ein vorhandenes Element wird nie als „nicht vorhanden“ gemeldet.

\subsection{Trade-off}
Bloomfilter balancieren drei zentrale Faktoren. Neben der Reject-Time (Zeit zur Ablehnung von Nicht-Mitgliedern) und dem benötigten Speicherplatz, die auch in konventionellen Hashing-Methoden berücksichtigt werden müssen, wird hier auch die erlaubte Fehlerrate betrachtet. Der zentrale Trade-off ist hierbei der akzeptable Anteil an False-Positives gegen die Speichereffizienz. Dieser ist bei der Implementierung eines Bloomfilters individuell konfigurierbar. 

Durch die kontrollierte Fehlerwahrscheinlichkeit wird der Speicherbedarf bedeutend reduziert, da dieser nicht von der Länge der Daten abhängt, sondern immer gleich viele Bits pro Element beträgt. Je niedriger die Fehlerrate gewählt ist, desto mehr Bits pro Element werden benötigt.
Bloomfilter sind besonders hilfreich, wenn die Mehrheit der Anfragen nicht-existente Elemente betrifft – hier liefern sie schnell ein definitives „Nein“ auf die Membership-Frage.

\section{Funktionsweise und Mathematische Grundlagen}
\subsection{Aufbau}
Hier stellst du die relevanten Theorien und Konzepte vor, die für dein Thema wichtig sind. Du erklärst zentrale Begriffe und gibst einen Überblick über den aktuellen Forschungsstand.

Nach \cite{Mustermann2023} ist dies ein wichtiger Aspekt der Forschung. Weitere Studien zeigen, dass...

\subsection{Einfügen/Suchen}
Eine kritische Auseinandersetzung mit der existierenden Literatur. Was wurde bereits erforscht? Welche Lücken gibt es?

\subsection{Formeln zur Evaluierung}
\section{Pseudocode und Implementierung}
Ein Bloom-Filter lässt sich mit drei grundlegenden Operationen beschreiben: Initialisierung, Einfügen und Abfragen.

\subsection{Initialisierung}
Bei der Initialisierung werden alle $m$ Bits im Array auf $0$ gesetzt und $k$ Hash-Funktionen festgelegt. Je kleiner die gewünschte Fehlerrate ist, desto größer muss $m$ gewählt werden.

\begin{algorithm}
\caption{Initialisierung eines Bloom-Filters}
\begin{algorithmic}[1]
\State Erzeuge Bit-Array $B[0 \dots m-1]$ und setze alle Bits auf $0$
\State Definiere Hash-Funktionen $h_1, h_2, \dots, h_k$
\end{algorithmic}
\end{algorithm}

\subsection{Einfügen}
Beim Einfügen wird für jedes Element $x$ eine Schleife genau $k$-mal ausgeführt. In jeder Iteration wird mithilfe der jeweiligen Hash-Funktion $h_i$ ein Index berechnet und das entsprechende Bit im Bitarray auf $1$ gesetzt. Der Modulo-Operator stellt sicher, dass der berechnete Index immer innerhalb des gültigen Bereichs von $0$ bis $m-1$ liegt. \cite{bloomJambura}

\begin{algorithm}
\caption{Einfügen eines Elements $x$}
\begin{algorithmic}[1]
\For{$i = 1$ \textbf{to} $k$}
    \State $index \gets h_i(x) \bmod m$
    \State $B[index] \gets 1$
\EndFor
\end{algorithmic}
\end{algorithm}


\subsection{Abfragen}
Für eine Abfrage werden dieselben $k$ Hash-Werte berechnet und die entsprechenden Positionen im Array überprüft. Existiert mindestens ein Bit mit dem Wert 0, kann man mit absoluter Sicherheit sagen, dass das Element nicht enthalten ist – es gibt keine False Negatives. Sind hingegen alle $k$ Bits gleich 1, gilt das Element als wahrscheinlich enthalten. Diese probabilistische Aussage ist das zentrale Merkmal des Bloom-Filters: Es sind False Positives möglich.

\begin{algorithm}
\caption{Abfrage eines Elements $x$}
\begin{algorithmic}[1]
\For{$i = 1$ \textbf{to} $k$}
    \State $index \gets h_i(x) \bmod m$
    \If{$B[index] = 0$}
        \State \Return \textbf{FALSE}
    \EndIf
\EndFor
\State \Return \textbf{TRUE}
\end{algorithmic}
\end{algorithm}

\section{Komplexitätsanalyse}

\subsection{Zeitkomplexität}

Sowohl das Einfügen als auch das Abfragen eines Elements haben eine Zeitkomplexität
von $\mathcal{O}(k)$, wobei $k$ die Anzahl der Hash-Funktionen bezeichnet.
Entscheidend ist dabei, dass diese Zeit \emph{unabhängig} von der Anzahl~$n$
der bereits im Filter gespeicherten Elemente ist. Der Grund
dafür liegt in der Struktur des Filters: Es werden keine Elemente explizit
gespeichert, sondern lediglich Bits in einem Array der Größe~$m$ gesetzt oder
gelesen. Egal ob sich 1.000 oder 100~Millionen Elemente im Filter befinden
-- die Abfragezeit bleibt konstant~\cite{burtonbloom}.

\subsection{Speicherkomplexität}

Die Speicherkomplexität beträgt $\mathcal{O}(m)$, wobei $m$ die Größe des
Bit-Arrays ist. Im Gegensatz zu klassischen Datenstrukturen hängt dieser
Speicherbedarf \emph{nicht} von der Größe der gespeicherten Elemente ab,
sondern nur von zwei Faktoren: der Anzahl der zu speichernden Elemente~$n$
und der akzeptierten False-Positive-Rate~$\varepsilon$~\cite{burtonbloom}.

Als praktische Faustregel gilt: Bei einer Fehlerrate von etwa $1\,\%$ benötigt
ein Bloom-Filter weniger als $10$~Bits pro Element. Das ist
bemerkenswert effizient -- unabhängig davon, ob es sich bei den Elementen um
kurze Zeichenketten oder lange URLs handelt~\cite{fan2000}.

\subsection{Vergleich mit anderen Datenstrukturen}

Tabelle~\ref{tab:vergleich} stellt die Komplexitätseigenschaften des
Bloom-Filters denen einer Hash-Tabelle mit verketteten Listen sowie eines
balancierten Baums gegenüber.

\begin{table}[ht]
    \centering
    \begin{tabular}{p{3.5cm}p{3.5cm}p{3.5cm}p{3.5cm}}
        \hline
        \textbf{Eigenschaft} & \textbf{Bloom-Filter} & \textbf{Hash-Tabelle mit Chaining} & \textbf{Balancierter Baum} \\
        \hline
        Zeitkomplexität & $\mathcal{O}(k)$ & $\varnothing\,\mathcal{O}(1)$, worst $\mathcal{O}(n)$ & $\mathcal{O}(\log n)$ \\
        Speicherkomplexität & $\mathcal{O}(m)$ & $\mathcal{O}(n)$ & $\mathcal{O}(n)$ \\
        Genauigkeit & Probabilistisch & Exakt & Exakt \\
        \hline
    \end{tabular}
    \caption{Vergleich ausgewählter Datenstrukturen}
    \label{tab:vergleich}
\end{table}


Die \textbf{Hash-Tabelle mit verketteten Listen} erreicht im Durchschnitt
$\mathcal{O}(1)$ für Einfüge- und Suchoperationen, kann im schlechtesten Fall
jedoch auf $\mathcal{O}(n)$ anwachsen. Da jedes Element explizit gespeichert
wird, beträgt der Speicherbedarf $\mathcal{O}(n)$ -- typischerweise
$64$~Bits oder mehr pro Element (Nutzdaten plus Pointer). Der wesentliche
Vorteil liegt in der Exaktheit: Es gibt keine False Positives, und Elemente
können jederzeit wieder abgerufen werden~\cite{Broder01012004}.

Der \textbf{balancierte Baum} hat eine Zeitkomplexität von $\mathcal{O}(\log n)$
für Suche und Einfügen. Die Laufzeit steigt mit wachsender Elementanzahl
langsam an, da bei jedem Schritt etwa die Hälfte der verbleibenden Elemente
verworfen wird. Der Speicherbedarf ist ebenfalls $\mathcal{O}(n)$. Der Vorteil
liegt in der Möglichkeit, Elemente geordnet zu speichern, was zusätzliche
Operationen wie Bereichsabfragen erlaubt~\cite{Broder01012004}.

\subsection{Speichereffizienz in der Praxis}

Um die Speicherersparnis greifbar zu machen, betrachten wir ein konkretes
Beispiel: Für $100$~Millionen URLs benötigt ein Bloom-Filter bei einer
Fehlerrate von $1\,\%$ rund $120$~Megabyte. Eine Hash-Tabelle mit denselben
Einträgen würde hingegen über ein Gigabyte beanspruchen. Das
ist nicht nur ein quantitativer, sondern oft ein qualitativer Unterschied --
nämlich der zwischen einem System, das auf einem Endgerät lauffähig ist, und
einem, das einen dedizierten Server erfordert.

Dieser enorme Vorteil hat allerdings seinen Preis: Ein Bloom-Filter beantwortet
ausschließlich die Frage \emph{,,Ist das Element möglicherweise in der
Menge? ``} Er kann weder Elemente aufzählen noch löschen, noch gibt er die Elemente selbst zurück~\cite{bloomJambura}.
\section{Probleme von Bloom-Filtern und Lösungen}
\subsection{Das Löschen von Elementen}
Der klassische Bloom-Filter besitzt unter anderem die Einschränkung, dass er das Löschen von Elementen nicht unterstützt. Möchte man ein Element entfernen, liegt es zunächst nahe, die entsprechenden Bits im Bit-Array wieder auf 0 zu setzen. Genau hier entsteht jedoch ein fundamentales Problem. Mehrere Elemente können auf dieselbe Position im Bit-Array hashen. Wird ein Bit zurückgesetzt, entfernt man daher nicht nur das gewünschte Element, sondern gleichzeitig auch alle anderen Elemente, die an dieser Position gespeichert wurden. Das eigentliche Element ist zwar entfernt, aber andere Elemente gelten nun ebenfalls als nicht mehr vorhanden, obwohl sie eigentlich noch im Filter sein sollten.

Eine Lösung für dieses Problem ist der Counting Bloom Filter. \cite{bloomThiele} Das Grundprinzip beim Einfügen bleibt dabei gleich wie beim klassischen Bloom-Filter. Der Unterschied besteht darin, dass man an jeder Position nicht nur ein einzelnes Bit speichert, sondern einen kleinen Zähler. Dieser Zähler wird beim Einfügen eines Elements um 1 erhöht und beim Löschen wieder um 1 verringert. Typischerweise sind diese Zähler 4 Bit groß und können somit Werte von 0 bis 15 speichern. Dadurch wird es möglich, Elemente sicher zu löschen, ohne andere Einträge unbeabsichtigt zu beeinflussen.

Allerdings hat der Counting Bloom Filter auch Nachteile. Der Speicherverbrauch ist deutlich höher, da statt eines einzelnen Bits nun 4 Bits pro Position benötigt werden. Bei gleicher Genauigkeit benötigt ein Counting Bloom Filter somit ungefähr das Drei- bis Vierfache an Speicher im Vergleich zum klassischen Bloom-Filter. 
Außerdem können die Zähler überlaufen. Wenn mehr als 15 Elemente auf dieselbe Position hashen, reichen 4 Bits nicht mehr aus. Man könnte größere Zähler verwenden, allerdings würde das den Speicherbedarf weiter erhöhen. 
Der Counting Bloom Filter eignet sich daher besonders dann, wenn häufig gelöscht werden muss - man bezahlt diese Möglichkeit jedoch mit einem deutlich höheren Speicherverbrauch. An Verbesserungen wird zwar gearbeitet, doch eine genauere Betrachtung würde den Rahmen dieser Arbeit sprengen.\cite{CountingThiele}

\subsection{Größenplanung}
Ein weiteres grundlegendes Problem klassischer Bloom-Filter ist die Größenplanung. In der Regel muss man vorher festlegen, wie groß der Filter sein soll. Ist er zu klein dimensioniert, steigt die Fehlerwahrscheinlichkeit stark an. Die Bits werden sehr schnell gesetzt und die False-Positive-Rate nimmt deutlich zu. Ist der Filter hingegen zu groß gewählt, wird Speicherplatz verschwende, da möglicherweise Kapazitäten reserviert werden, die nie vollständig genutzt werden.

Der Scalable Bloom Filter bietet hier eine Lösung durch dynamisches Wachstum. \cite{bloomThiele}Er besteht aus mehreren klassischen Bloom Filtern, die nacheinander erstellt werden. Sobald ein Filter eine bestimmte Auslastung erreicht, wird ein neuer, größerer Filter mit eineer strengeren Fehlerrate hinzugefügt. Auf diese Weise bleibt die Gesamtfehlerwahrscheinlichkeit über alle Filter hinweg kontrollierbar. Selbst wenn mehrere Filter hinzukommen, bleibt die kombinierte Fehlerrate in akzeptablen Grenzen. 
Der große Vorteil ist, dass der Filter beliebig wachsen kann, ohne komplett neu aufgebaut werden zu müssen. Ein Nachteil ist jedoch, dass Abfragen mit jedem zusätzlichen Filter etwas langsamer werden, da mehrere Filter überprüft werden müssen.
Neben Counting- und Scalable-Varianten gibt es noch viele weitere spezielle Varianten von Bloom-Filtern, die jedoch den Rahmen dieser Arbeit überschreiten würden.\cite{ScalableThiele}

\section{Cuckoo Filter}
Eine alternative Datenstruktur stellt der Cuckoo Filter dar. Hierbei handelt es sich nicht merh wirklich um einen Bloom-Filter, dennoch verfolgt er dasselbe Ziel: speichereffiziente Mengenabfragen bei geringen Fehlerraten. 
Der Cuckoo Filter basiert nicht auf einem Bit-Array, sondern auf einer Hash-Tabelle mit kleinen Fächern, sogenannten Buckets. In diesen Buckets werden Fingerabdrücke, sogenannte Fingerprints, gespeichert. Das sind kurze, eindeutige Kennungen der Elemente mit nur wenigen Bits Länge.

Beim Einfügen eines Elements wird mithilfe einer Hash-Funktion berechnet, in welches Fach es gehört. Jedes Element besitzt dabei genau zwei mögliche Buckets, in denen es abgelegt werden kann. Ist in einem dieser Buckets noch Platz vorhanden, wird der Fingerprint dort gespeichert. Sind jedoch beide Buckets belegt, greift das sogenannte Cuckoo-Prinzip. Hier verdrängt das neue Element einen bestehenden Eintrag aus einem der beiden Buckets. Das verdrängte Element muss sich anschließend einen neuen Platz in seinem alternativen Bucket suchen. Dieser Prozess kann sich fortsetzen, bis schließlich alle Elemente einen Platz gefunden haben.

Der Cuckoo Filter bringt sowohl Vorteile als auch Nachteile mit sich. Ein großer Vorteil ist, dass Elemente problemlos gelöscht werden können, da die Fingerprints direkt gespeichert sind und gezielt entfernt werden können. 
Außerdem sind Abfragen sehr schnell, da nur zwei Buckets geprüft werden müssen.
Ein Nachteil zeigt sich bei sehr hoher Auslastung der Hash-Tabelle. In solchen Fällen kann die Verdrängungskette sehr lang werden, ohne dass ein freier Platz gefunden wird. Dann muss die gesamte Struktur vergrößert werden. 
Studien zeigen jedoch, dass Cuckoo Filter in vielen realen Anwendungen praktisch besser abschneiden als klassische Bloom-Filter.\cite{CuckooThiele}
Klassische Bloom-Filter sind dennoch besonders sinnvoll, wenn sehr große Datenmengen verarbeitet werden, der verfügbare Speicher knapp oder teuer ist, kleine Fehlerraten akzeptiert werden können und sie als Vorfilter von aufwendigen oder rechenintensiven Operationen eingesetzt werden. \cite{Broder01012004}
\section{Anwendungsbeispiele}
\subsection{Web-Proxy-Caching}
In verteilten Peer-to-Peer Netzwerken arbeiten mehrere Proxy-Server zusammen und tauschen sich untereinander aus. Bei einer Anfrage nach einer Webseite sucht ein Proxy zunächst im eigenen Cache, ob er diese bereits gespeichert hat. Wenn das nicht der Fall ist, spricht man von einem Cache-Miss und es wird gecheckt, ob sich die Webseite im Cache eines anderen Proxys befindet. Wird sie hier gefunden, wird die Anfrage an den entsprechenden Proxy weitergeleitet, anstatt die Seite direkt aus dem Web zu laden.

Damit dieses System funktioniert, muss jeder Proxy über den Inhalt der Caches aller anderen Proxies Bescheid wissen. Um den enormen Netzwerkverkehr, der beim wiederholten Austausch der kompletten URL-Listen entstehen würde, zu vermeiden, kommen hier Bloomfilter zum Einsatz. In dem Summary Cache Protokoll tauschen Proxies periodisch Bloomfilter untereinander aus, die den Inhalt ihres Caches zusammenfassen. Wenn nun ein Cache-Miss auftritt, werden die Bloomfilter jener anderen Proxies konsultiert, die ein positives Ergebnis versprechen und die Anfrage wird entsprechend weitergeleitet. 

Hierbei können False-Positives auftreten, wobei es dann zu einer minimalen Verzögerung kommt. Die massive Reduktion des Netzwerkverkehrs durch den Bloomfilter überwiegt diesen Nachteil bei Weitem. Das Summary Cache Protokoll wird beispielsweise im Web-Proxy-Cache „Squid“ eingesetzt.

\subsection{Google Bigtable}
Bloomfilter werden oft in Datenbanksystemen verwendet, wobei Google Bigtable ein bekanntes Beispiel hierfür ist. Bigtable speichert die Daten auf der Festplatte in Sorted-String-Tables (SSTables). Wenn eine Leseoperation durchgeführt werden soll, müssen potenziell mehrere dieser Tables durchsucht werden, bis die gewünschten Daten gefunden werden. Da jede Table auf der Festplatte liegt, verursacht jeder Zugriff auf eine SSTable auch einen teuren Festplattenzugriff. Besonders problematisch im Bezug auf die benötigten Ressourcen wird dies bei Abfragen nach nicht-existenten Daten.

Kommen jetzt die Bloomfilter zum Einsatz, ändert sich dies drastisch. Für jede SSTable wird ein Bloomfilter im Hauptspeicher gehalten, der Auskunft über deren Inhalt gibt. Vor einem Festplattenzugriff wird also der Filter befragt, ob die gesuchten Daten in der Table enthalten sind. Bei einem positiven Ergebnis wird der Zugriff durchgeführt, ansonsten kann er eingespart werden. 

\subsubsection{Beispiel Anfrage}
Angenommen es wird eine Anfrage auf den Schlüssel $X$ gestellt und auf der Festplatte liegen drei SSTables. Ohne Verwendung von Bloomfiltern müssten alle drei Tables abgerufen und durchsucht werden, also drei Festplattenzugriffe druchgeführt werden.
Unter Einsatz von Bloomfiltern werden jedoch zuerst diese konsultiert. Filter 1 könnte beispielsweise melden, dass Schlüssel $X$ definitiv nicht in SSTable 1 vorhanden ist, dann kann diese übersprungen werden. Filter 2 meldet jetzt das gleiche für SSTable 2, also wird diese auch übersprungen. Filter 3 sagt jetzt, dass sich $X$ in Table 3 befinden könnte – der Zugriff wird durchgeführt. 
Demzufolge wurde nur ein Festplattenzugriff durchgeführt, bis der gesuchte Schlüssel $X$ gefunden wurde, das bedeutet eine Ersparnis von zwei Zugriffen durch die Verwendung von Bloomfiltern.

\subsection{Weitere Anwendungen}
Heute kommen Bloomfilter in zahlreichen Systemen zum Einsatz. Google Chrome nutzt sie für Safe-Browsing zur Malware-Erkennung. Verschiedene Sicherheitsdienste, unter anderem „Have I Been Pwned“, prüfen mit ihrer Hilfe, ob Passwörter kompromittiert wurden, ohne dabei die komplette Leak-Datenbank lokal speichern zu müssen. Neben Google Bigtable setzen auch weitere Datenbanksysteme, wie Apache Cassandra und LevelDB auf die Vorteile von Bloomfiltern, um unnötige Festplattenzugriffe zu vermeiden.

\begin{abstract}
Bloomfilter sind probabilistische Datenstrukturen, die zur effizienten Lösung des Membership-Problems entwickelt wurden. Sie basieren auf einem Bit-Array und mehreren Hashfunktionen, wodurch eine sehr hohe Speicher- und Zeiteffizienz erreicht wird, jedoch mit einer geringen, konfigurierbaren Wahrscheinlichkeit für False-Positive-Ergebnisse. Die Fehlerrate hängt dabei von Parametern wie der Größe des Bit-Arrays, der Anzahl der Hashfunktionen und der Anzahl der gespeicherten Elemente ab und stellt einen Trade-off zwischen Genauigkeit und Speicherbedarf dar. Einschränkungen wie das fehlende Löschen von Elementen und die feste Größenplanung können durch Erweiterungen wie Counting Bloom Filter oder Scalable Bloom Filter verbessert werden, während der Cuckoo Filter eine alternative Lösung darstellt. Aufgrund ihrer Effizienz kommen Bloom-Filter unter anderem in Bereichen wie Web-Caching und Datenbanksystemen zum Einsatz.
\end{abstract}

\newpage

\bibliography{literatur}
\bibliographystyle{unsrtnat}

\end{document}